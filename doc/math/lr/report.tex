\documentclass[fleqn,oneside,12pt]{article}
\date{}
\newcommand{\up}[1]{a_{#1}^{\dagger}}
\newcommand{\down}[1]{a_{#1}}
\newcommand{\ud}{\mathrm{d}}
\newcommand{\EE}{\mathrm{E}}
\newcommand{\diff}[2]{\frac{\ud {#1}}{\ud {#2}}}
\newcommand{\pdiff}[2]{\frac{\partial #1}{\partial #2}}
\newcommand{\fl}{\mathrm{fl}}
\newcommand{\sqrtfrac}[2]{\sqrt{\frac{#1}{#2}}}
\newcommand{\ui}{\mathbf{i}}
\newcommand{\uj}{\mathbf{j}}
\newcommand{\ua}{\mathbf{a}}
\newcommand{\ub}{\mathbf{b}}
\newcommand{\bi}{\bar{i}}
\newcommand{\bj}{\bar{j}}
\newcommand{\ba}{\bar{a}}
\newcommand{\bb}{\bar{b}}
\newcommand{\tb}{\phantom{}^{Q} \hat b}
\newcommand{\erf}{\mathrm{erf}}
\newcommand{\erfc}{\mathrm{erfc}}
\setlength{\parskip}{1ex plus 0.5ex minus 0.2ex}
\setlength{\parindent}{0pt}
\usepackage[top=1in, bottom=1in, left=1in, right=1in]{geometry}
\usepackage{amsmath}
\usepackage[super]{cite}
\usepackage{amssymb}
\usepackage[usenames,dvipsnames]{color}
\usepackage{verbatim}
\usepackage{fancyhdr}
\usepackage{graphicx}
\pagestyle{fancyplain}
\begin{document}
\lhead{Low-Rank Quantities in Lightspeed}
\rhead{Ramya Rangan and Rob Parrish}

\subsection{Laplace Denominators}

\subsubsection{Approximation of $1/x$ on $[0, \infty)$ by Exponential Sums}

A very useful identity is,
\[
\frac{1}{x}
=
\int_{0}^{\infty}
\mathrm{d} \alpha \
e^{-x \alpha}
\approx
\sum_{w}
\omega_{w}
e^{-\alpha_{w} x}
\]
The last line indicates that a quadrature rule $\{ <\alpha_{w}, \omega_{w} > \}$
is being used to approximate the integral over $\alpha$. This understanding of
the Laplace resolution is due to Almlof,\footnote{J. Alml\"of, \emph{Chem. Phys.
Lett.}, \textbf{181}, 319 (1991)} who explored a variety of quadrature rules to
approximate the integral. 

More recently, Braess and Hackbusch have realized that it is better to directly
fit $1/x$ as a sum of exponentials for $x$ in the range of $[1, R)$,\footnote{D.
Braess and W. Hackbusch, \emph{IMA J. Numer. Anal.}, \textbf{25}, 685 (2005)}
rather than using the quadrature interpretation,
\[
\frac{1}{x}
\approx
\sum_{w}
\omega_{w}
e^{-\alpha_{w} x}
:
\left .
\left |
\frac{1}{x}
-
\sum_{w}
\omega_{w}
e^{-\alpha_{w} x}
\right |
<
\delta
\right |_{x \in [1, R)}
\]
He has located the optimal quadrature rule by the Remez minimax algorithm for a
variety of $R$ and for $n_{w}$ between 1 and 50, and has published the
quadrature rules $\{ < \alpha_{w}, \omega_{w} > \}$ and corresponding maximum
error $\delta$ for each case.\footnote{See Hackbusch's website.}

\subsubsection{Laplace Denominators in Electronic Structure Theory}

We start from a set of ``occupied'' orbitals labeled by $i,j,k,l$ with orbital
energies $\{ \epsilon_{i} \}$ and a set of ``virtual'' orbitals labeled by
$a,b,c,d$ with orbital energies $\{ \epsilon_{a} \}$. We assert that the
occupied orbital energies are strictly less than the virtual orbital energies,
i.e., $\epsilon_{i} < \epsilon_{a} \ \forall \ i,a$.

A second-order orbital energy denominator has the form,
\[
\Delta_{ijab}
\equiv
\frac{1}{
\epsilon_{a} 
+
\epsilon_{b} 
-
\epsilon_{i} 
-
\epsilon_{j} 
}
\approx
\bar \tau_{i}^{w}
\bar \tau_{j}^{w}
\tau_{a}^{w}
\tau_{b}^{w}
\]
A third-order orbital energy denominator has the form,
\[
\Delta_{ijkabc}
\equiv
\frac{1}{
\epsilon_{a} 
+
\epsilon_{b} 
+
\epsilon_{c} 
-
\epsilon_{i} 
-
\epsilon_{j} 
-
\epsilon_{k} 
}
\approx
\bar \tau_{i}^{w}
\bar \tau_{j}^{w}
\bar \tau_{k}^{w}
\tau_{a}^{w}
\tau_{b}^{w}
\tau_{c}^{w}
\]
In an $n$-th order orbital energy denominator, we define the ``extent'' of the
argument $x$ to be,
\[
R 
\equiv
\frac{
\max(\epsilon_{a})
-
\min(\epsilon_{i})
}{
\min(\epsilon_{a})
-
\max(\epsilon_{i})
}
\]
Using this and a user-specified error criterion $\delta$, we select one of the
Hackbush quadratures $\{ < \alpha_{w}, \omega_{w} > \}$ for use on $[1, R)$. We
now define a scaling constant,\footnote{What we are doing here is to write
\[
\frac{1}{\eta}
=
\gamma
\frac{1}{\eta / \gamma}
\]
with $\gamma$ chosen so that the maximum argument of the second denominator is
$1$. This simple rescaling then lets us use Hackbusch's quadrature rules.}
\[
\gamma
\equiv
n
(
\min(\epsilon_{a})
-
\max(\epsilon_{i})
)
\]
We now define the modified quadrature rule,
\[
\alpha_{w}'
\equiv
\alpha_{w}
/
\gamma
,
\
\omega_{w}'
\equiv
\omega_{w}
/
\gamma
\]
And arrive at, e.g.,
\[
\Delta_{ijab}
\approx
\sum_{w}
\omega_{w}'
\exp(- \alpha_{w}' (
\epsilon_{a} 
+
\epsilon_{b} 
-
\epsilon_{i} 
-
\epsilon_{j} 
))
\]
Finally, by inspection,
\[
\bar \tau_{i}^{w}
\equiv
\sqrt[2n]{\omega_{w}'}
e^{+\alpha_{w}' \epsilon_{i}}
\]
\[
\tau_{a}^{w}
\equiv
\sqrt[2n]{\omega_{w}'}
e^{-\alpha_{w}' \epsilon_{a}}
\]

Note that whereas Hackbush's quadrature rules bound the maximum \emph{absolute}
error of $1/x$ to $\delta$ for $x \in [1, R)$, our choice of quadrature
selection and scaling bounds the maximum \emph{relative} error of
$\Delta_{ijab}$ to $\delta$ for the given orbital eigenvalues.

\section{Density Fitting}

A key quantity in electronic structure theory is the electron repulsion
integral (ERI),
\[
(pq|rs)
\equiv
\iint_{\mathbb{R}^6}
\mathrm{d}^3 r_1 \
\mathrm{d}^3 r_2 \
\phi_{p} (\vec r_1)
\phi_{q} (\vec r_1)
\frac{1}{r_{12}}
\phi_{r} (\vec r_2)
\phi_{s} (\vec r_2)
\]
Here $\{ \phi_{p} (\vec r_1) \}$ is the atomic-orbital basis of atom-centered
Gaussian functions (contracted Cartesian or real solid harmonic basis
functions). Analytical formulae exist for the ERIs, but they are a major
bottleneck to generate, store, and manipulate.

In density fitting,\footnote{Specifically robust Coulomb-metric density
fitting.} we approximate the electron repulsion integrals
\[
(pq|rs)
\approx
(pq|A)
(A|B)^{-1}
(B|rs)
=
(pq|A)
(A|C)^{-1/2}
(C|B)^{-1/2}
(B|rs)
\equiv
L_{pq}^{C}
L_{rs}^{C}
\]
Here $\{ \chi_{A} (\vec r_1) \}$ is an auxiliary basis set of atom-centered
Gaussian functions. This is usually defined to accompany a given primary AO
basis set, and has typically $2-5\times$ the number of basis functions as the
primary basis.

In the above, the two-index integrals (often called the DF metric) are,
\[
(A|B)
\equiv
\iint_{\mathbb{R}^6}
\mathrm{d}^3 r_1 \
\mathrm{d}^3 r_2 \
\chi_{A} (\vec r_1)
\frac{1}{r_{12}}
\chi_{B} (\vec r_2)
\]
and the three-index integrals are,
\[
(A|pq)
\equiv
\iint_{\mathbb{R}^6}
\mathrm{d}^3 r_1 \
\mathrm{d}^3 r_2 \
\chi_{A} (\vec r_1)
\frac{1}{r_{12}}
\phi_{p} (\vec r_1)
\phi_{q} (\vec r_2)
\]
If one has formulae (or codes) for the four-center integrals $(pq|rs)$, one can
very easily obtain the two- and three-center integrals by substituting ``null''
basis functions $\varnothing (\vec r_1) \equiv 1$ (a single Gaussian $s$
function at the origin with exponent of $0$ and prefactor of $1$) into the
expressions for the four-center integrals, e.g., $(A|B) \equiv (A \varnothing |
B \varnothing)$ or $(A|pq) \equiv (A \varnothing | pq)$. The Cauchy-Schwarz
bound can be exploited to only form (and, optionally, only store) significant
$pq$ pairs in the three-index integrals above.\footnote{The LS \texttt{PairList}
object tells you about the significant shell pairs. We throw away whole shell
pairs to preserve rotational invariance.}

The form seen above in terms of $L_{pq}^{C}$ is the ``symmetric'' form of the DF
approximation, and involves applying the inverse square root of the metric to
the three-index integrals. In practice, the auxiliary basis may be rather
ill-conditioned in the Coulomb metric, but this is easily dealt with by
discarding the smallest eigenvectors in the metric to preserve an inverse
relative condition number of $\kappa$ in the basis,
\[
(A|B)^{-1/2}
\equiv
U_{AC}
s_{C}^{-1/2}
U_{BC}
: \
s_{C} > \kappa s_{C}^{\mathrm{max}}
,
\
(A|B)
=
U_{AC}
s_{C}
U_{BC}
\]
Here $U_{AC}$ are the eigenvectors and $S_{C}$ are the eigenvalues of the metric
$(A|B)$. The value of $\kappa$ is specified by the user, and should be small
enough that significant parts of the auxiliary basis are not discarded, but
large enough that numerical precision artifacts do not affect the result.
$\kappa = 10^{-12}$ seems to be a reasonable choice across a vast range of
EST problems, provided that double precision is used.

\subsection{AO Density Fitting}

Our objective in \texttt{ao\_df} is to produce the tensor,
\[
L_{pq}^{C} \equiv (C|A)^{-1/2} (A|pq)
\]
The tensor should be stored as $Apq$, and will not exploit spatial sparsity or
permutational symmetry. 

\begin{enumerate}
\item{Form the metric matrix $(C|A)$.}
\item{Form the metric inverse square root $(C|A)^{-1/2}$ via the
\texttt{Tensor::power} method. Be sure to condition the inverse via the
user-supplied inverse relative condition number $\kappa$.}
\item{Allocate a single tensor \texttt{L} with indices $Apq$.}
\item{Form the integrals $(A|pq)$ and place into \texttt{L}. Do not compute
insignificant integrals}
\item{Perform the matrix multiplication $(C|A)^{-1/2} (A|pq)$ in chunks of $pq$
(using chunks of $N_A$ $pq$ indices works great). Copy the chunk from \texttt{L}
into a temporary buffer, and then \texttt{dgemm} it back into \texttt{L}.}
\end{enumerate}
Overall, this uses a single copy of size $Apq$, plus two copies of $(C|A)$. And
it is fairly simple as such algorithms go. E.g., we could have avoided a memcpy
in the last stage by directly computing the integrals in the \texttt{dgemm}
buffer. But then we would have had issues with shell granularity of the
integrals. 

\subsection{MO Density Fitting}

Our objective in \texttt{mo\_df} is to produce the tensor,
\[
L_{ia}^{C} \equiv (C|A)^{-1/2} (A|pq) C_{pi} \bar C_{pa}
\]
The tensor should be stored as $iaC$. A number of combinations of $C_{pi}$ and
$\bar C_{qa}$ should be allowed. The transformation in $C_{pi}$ should be
performed first, and the documentation should recommend using the smaller
(usually occupied) index for the first part of the transformation, to save
FLOPS/storage. 

\begin{enumerate}
\item{Form the metric matrix $(C|A)$.}
\item{Form the metric inverse square root $(C|A)^{-1/2}$ via the
\texttt{Tensor::power} method. Be sure to condition the inverse via the
user-supplied inverse relative condition number $\kappa$.}
\item{Allocate a temporary copy of the target tensors with indices $Aia$}
\item{For stripes of $A$ (up to say 2 GB, using whole $A$ shells):}
\begin{enumerate}
\item Form the integrals $(A|pq)$ for the stripe. Do not compute insignificant
integrals.
\item Perform the contraction $(A|pi) = (A|pq) C_{qi}$ with one \texttt{dgemm}
call for the stripe.
\item Perform the contraction $(A|ia) = (A|pi) \bar C_{pa}$ with
$N_{A}^{\mathrm{stripe}}$ \texttt{dgemm} calls. The result should be placed
directly in the temporary target tensors.
\end{enumerate}
\item Perform the fitting $L_{ia}^{C} = (C|A)^{-1/2} (A|ia)$ for each target
tensor. The striping will reverse from $Aia$ to $iaC$ in this step. An extra
copy of size $iaC$ can be used in this step. 
\end{enumerate}
Here the integrals $(A|pq)$ can be used for all combinations of $C_{pi}$ and
$\bar C_{qa}$. If one is clever, the first-half transform in 4b can be re-used
for tasks with the same $C_{pi}$ tensor. 


\end{document}
